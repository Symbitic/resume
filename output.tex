\documentclass{mcdowellcv}

\usepackage{amsmath}
\usepackage{hyperref}
\usepackage{xcolor}
\usepackage{fontawesome}
\definecolor{cyan}{HTML}{00FFFF}

\hypersetup{
  colorlinks=true,
  linkcolor=blue,
  urlcolor=blue,
  % pdfnewwindow=true,
  pdftitle={Alexander L. Shaw's Resume},
}
\urlstyle{same}

\newcommand{\email}[3][blue]{\href{#2}{\color{#1}{#3}}}

% Set applicant's personal data for header
\name{Alexander L. Shaw}
\address{
  103 G Street SW, B214\linebreak
  Washington
  D.C.,
  20024
}
\contacts{
  {{\faPhone}~}(815) 590-3550\linebreak
  {\small\faEnvelopeO} \email{mailto:alex.shaw.as@gmail.com}{alex.shaw.as@gmail.com}\linebreak
  {{\faGlobe}~}\url{https://alshaw.io}\linebreak
  {{\faGithub}~}\href{https://github.com/Symbitic}{Symbitic}
}

\begin{document}

  % Print the header
  \makeheader

  % Print the content
    \begin{cvsection}{Education}
        \begin{cvsubsection}
      {Bloomington, IL}
      {Illinois State University}
      {08/20/2012---05/15/2015, 08/14/2017---12/21/2017}
      \begin{itemize}
                \item \textbf{Major:} Information Technology --- IT Security
                \item \textbf{Coursework:} Calculus II; Databases; Java; C++; Cryptography;
Systems Development
              \end{itemize}
    \end{cvsubsection}
      \end{cvsection}
  
    \begin{cvsection}{Employment}
        \begin{cvsubsection}
      {IT Specialist (App Software)}
      {Census Bureau}
      {8/6/2018---present \linebreak 40 hours/week}
            
      \begin{itemize}
                \item Our branch is responsible for maintaining and updating the national
database of addresses.
                \item Regularly check, process, and merge incoming data from customers, such
as the USPS.
                \item Currently in the process of helping our branch transition to using Agile
and GitLab.
                \item Created csv2table to make transfering incoming data to tables for
manipulation much faster.
                \item Also created SQLDump, which performs the reverse---dumps the results of
a SQL query to a CSV file.
                \item Oral communication and problem-solving skills used on a daily basis.
                \item Successfully completed a training session in web scraping and machine
learning.
                \item \underline{Technologies used} includes Oracle 11g, PL/SQL, Java, RHEL,
Unix shell scripting.
              \end{itemize}
          \end{cvsubsection}
      \end{cvsection}
    
    \begin{cvsection}{Software Projects}
        \begin{cvsubsectiontext}{\textbf{Personal Website:} \url{https://alshaw.io} \textit{(for additional information and projects)}}
    \end{cvsubsectiontext}
          
        \begin{cvsubsection}{csv2table}{}{}
      \begin{itemize}
                \item Created for the Census Bureau.
                \item Our workflow required regularly loading CSV data from customers into an
OracleDB table.
                \item Regular solution was to use SQLLOADER, a cumbersome and unintuitive
process.
                \item csv2table was an easy-to-use tool that automated the process, making it
much faster to complete.
                \item Source code unavailable; proprietary and property of the Census Bureau.
                \item \underline{Utilized}: Java, JDBC, SQL, OracleDB.
              \end{itemize}
    \end{cvsubsection}
        \begin{cvsubsection}{Timeline of Terror}{}{}
      \begin{itemize}
                \item As a government worker, I was inspired to modernize the excellent
\href{https://en.wikipedia.org/wiki/The_Terror_Timeline}{Terror Timeline}
by Paul Thompson.
                \item Original used JavaServer Pages and MySQL. Stated cost was several
hundred dollars per-year.
                \item New version uses Gatsby, React, and GraphQL for server-side static
rendering. Total cost is \$12/year.
                \item Not endorsed, condoned, or approved by the Census Bureau or any other
government agency.
                \item \underline{Utilized}: Gatsby, React.js, Node.js, GraphQL, CSS.
              \end{itemize}
    \end{cvsubsection}
        \begin{cvsubsection}{markbook}{}{}
      \begin{itemize}
                \item Tool for writing books and complex technical documentation in Markdown.
                \item MVP in three days, feature parity with mdBook in two weeks.
                \item Extensive unit testing and integration testing.
                \item Features: PDF/HTML/ePub output, APA citations, bibliography.
                \item \underline{Utilized}: JavaScript, Markdown, Remark.js, Jest, Babel.
              \end{itemize}
    \end{cvsubsection}
      \end{cvsection}
    
    \begin{cvsection}{Skills}
    \begin{cvsubsection}{}{}{}
      \begin{itemize}
                \item \textbf{Programming Languages: }(\textit{expert}): C, C++, JavaScript, Bash, HTML/CSS
(\textit{familiar}): Java, SQL, COBOL (\textit{learning}): Python, Rust,
LaTeX
                \item \textbf{Software: }Git, Linux, Agile, Node.js, Oracle 11g, Visual Studio Code, MongoDB,
Qt5, Angular.js, React.js, Machine Learning, Amazon Web Services
              \end{itemize}
    \end{cvsubsection}
  \end{cvsection}
    
    \textbf{\footnotesize Schedule A and Contacts available upon request}
  \end{document}
